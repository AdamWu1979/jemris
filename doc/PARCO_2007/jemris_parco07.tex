\documentclass{nic-series}
\usepackage{graphicx}
\usepackage{palg}
\usepackage{palgorithm}
\parindent0pt

\newcommand{\epsfigure}[5]{
 \begin{figure}[#4!]
   \begin{center}
    \includegraphics[width=#5\columnwidth]{#1}
    \caption{{\sl #2}\label{#3}}
   \end{center}
 \end{figure}}

\newcommand{\epsfigures}[6]{
 \begin{figure}[#5!]
   \begin{center}
    \includegraphics[width=#6\columnwidth]{#1}
    \hspace*{0.2cm}
    \includegraphics[width=#6\columnwidth]{#2}
    \caption{{\sl #3}\label{#4}}
   \end{center}
 \end{figure}}

\newcommand{\epsfiguresv}[6]{
 \begin{figure}[#5!]
   \begin{center}
    \includegraphics[width=#6\columnwidth]{#1}
   \end{center}
   \begin{center}
    \includegraphics[width=#6\columnwidth]{#2}
    \caption{{\sl #3}\label{#4}}
   \end{center}
 \end{figure}}

\begin{document} 

\title{HPC Simulation of Magnetic Resonance Imaging}

\author{Tony~St\"ocker, Kaveh Vahedipour, \and N.~Jon~Shah }

\institute{Institute of Medicine,\\
         Research Centre J\"ulich, 52425 J\"ulich, Germany\\
         \email{\{t.stoecker, k.vahedipour, n.j.shah\}@fz-juelich.de}
          }

\maketitle

\begin{abstracts}
High performance computer (HPC) simulations provide helpful insights to the process of magnetic resonance
image (MRI) generation, e.g. for general pulse sequence design and optimization, artefact detection, validation of
experimental results, hardware optimization, MRI sample development and for education purposes. This manuscript presents
the recently developed simulator JEMRIS (Jülich Environment for Magnetic Resonance Imaging Simulation). JEMRIS is
developed in C++, the message passing is realised with the MPI library. The approach provides generally valid numerical
solutions for the case of classical MR spin physics governed by the Bloch equations. The framework already serves as a tool
in research projects, as will be shown for the important example of multidimensional spatially selective excitation.
\end{abstracts}

%%%%%%%%%%%%%%%%%%%%%%%%%%%%%%%%%%%%%%%%%%%%%%%%%%%%%%%%
\section{Introduction}
Bloch equation-based numerical simulation of MRI experiments is an essential tool for a variety of different research directions.
In the field of pulse sequence optimisation, e.g.~for artefact detection and elimination, simulations allow to
differentiate between effects related to principal physics and hardware imperfection. If the simulation environment
is further able to simulate the hardware malfunction, then results may be used for the optimisation of the hardware
itself. Another prominent application is the design of specialised RF pulses which is often based on numerical simulations
of the excitation process. In general, the interpretation and validation of experimental results benefit from comparisons
to simulated data, which is especially important in the context of MRI sample development, e.g.~for the development of implants.
Another important direction of application is image generation for the purpose of medical image processing -- here, the complete
control on the properties of the input data allows a tailored design of image processing algorithms for certain applications.\\
\ \\
The numerical simulation of an MRI experiment is in its most general form a demanding task. This is due to the fact that a huge
spin ensemble needs to be simulated in order to obtain realistic results. Therefore, several published approaches reduce the
problem size in different ways. The most prominent method is to consider only cases in which analytical solutions to the
problem exist \cite{benoit}. However, for the case of simultaneous RF excitation and time varying gradient fields no
analytical solution exists and, thus, the important field of selective excitation can not be studied with such an approach.
Apart from the computational demand, the complexity of the MRI imaging sequence is an obstacle. The difficulty of MRI
sequence implementation using the software environments of commercial MRI scanner vendors -- painfully experienced by many
researchers and pulse programmers -- can be significantly reduced with appropriate software design patterns.\\
\ \\
The JEMRIS project was initiated taking all the aforementioned considerations into account. It takes advantage of massive
parallel programming on HPC (high performance computing) architectures.
The aim of the project was to develop an MRI simulator which is general, flexible, and usable. In detail, it provides
a general numerical solution of the Bloch Equations on a huge ensemble of classical spins in order to realistically simulate
the MRI measurement process. Further, it takes various important off-resonance effects into account.

%%%%%%%%%%%%%%%%%%%%%%%%%%%%%%%%%%%%%%%%%%%%%%%%%%%%%%%%
\section{Theory}
\subsection{MR signal evolution}
%{\it MR signal \medskip evolution and MR image generation}\\
The present simulator is based on a classical description of MRI physics by means of the  Bloch equation, describing the sample by its
physical properties of equilibrium magnetisation, $M_0$, and the longitudinal and transverse relaxation times, $T_1$ and $T_2$, respectively.
It provides an exact description for the magnetisation vector, ${\bf M}({\bf r},t)$, of non-interacting spin-isochromates under the influence of
an external magnetic field. For MRI, the field decomposes to the strong static component, $B_0$, a temporal and spatial varying field along
the same direction (the imaging gradients ${\bf G}$) and the orthogonal components of the excitation radio-frequency (RF) field, $B_1$:
\begin{equation}\label{eq:field}
	{\bf B}({\bf r},t)\;=\;[B_0+{\bf G}(t)\cdot{\bf r}]{\bf e}_z+B_{1x}({\bf r},t){\bf e}_x+B_{1y}({\bf r},t){\bf e}_y\;.
\end{equation}
A mathematical and numerical treatment is greatly simplified in the rotating frame of reference, in which the effect of the main field is not seen
since the coordinate system rotates with the speed of the spin precession. Here, the formulation of the Bloch equation in cylindrical coordinates is
very well suited for a numerical implementation
\begin{small}
\begin{equation}\label{eq:bloch}
 \left ( \begin{array}{c} \dot{M}_r \\ \dot{\varphi} \\ \dot{M}_z \end{array} \right ) \;=\;
 \left ( \begin{array}{ccc}
 \cos\varphi & \sin\varphi & 0 \\ -\frac{\sin\varphi}{M_r} & \frac{\cos\varphi}{M_r} & 0 \\ 0 & 0 & 1
 \end{array} \right )\cdot
 \left [
 \left ( \begin{array}{ccc} -\frac{1}{T_2} & \gamma B_z & -\gamma B_y \\
                             -\gamma B_z &-\frac{1}{T_2} &  \gamma B_x \\
                             \gamma B_y & -\gamma B_x &-\frac{1}{T_1}
          \end{array} \right )
 \cdot \left ( \begin{array}{c} {M}_r\cos\varphi \\ {M}_r\sin\varphi \\ {M}_z \end{array} \right )
 + \left ( \begin{array}{c} 0 \\ 0 \\ \frac{M_0}{T_1} \end{array} \right ) \right ]
\end{equation}
\end{small}
\ \\
where $\gamma$ is the gyromagnetic ratio of the inspected nuclei (usually protons). The complex MR signal is then described by the signal equation
\begin{equation}\label{eq:signal}
 S(t)\;\propto\;\int_{V}\,M_r({\bf r},t)\exp[\mbox{i}\varphi({\bf r},t)]\;d^{3}r
\end{equation}
which integrates all components within the RF coil volume, $V$. For the description of the MR measurement process, the time
evolution of different spin isochromates is completely independent of each other and  the problem is ideally suited for a numerical treatment with
parallel processing. \\ In MRI, the measurement process is expressed by the MRI sequence \cite{haacke} which describes the timing of pulsed currents
in the various coils to induce the RF field for excitation and the gradient field for encoding spatial information to the phase of the MR signal: 
$\varphi({\bf r},t)\,=\,\gamma \int {\bf G}(t)\,dt\cdot{\bf r}\,\equiv\,{\bf k}(t)\cdot{\bf r}$. Inserting this expression in Eq.~(\ref{eq:signal}) shows
that the MR signal, $S(t)$, can be reordered as a function of the k-vector, $S({\bf k})$. Then, the MR image, $M_r({\bf r})$, is given by the Fourier
transformation of the acquired signal. Thus, the timing of the gradient pulses defines the so called k-space trajectory, ${\bf k}(t)$, along which the
necessary information for image generation is acquired during the measurement. In general the MR image,
$M_r({\bf r})=M_r(M_0({\bf r}),T_1({\bf r}),T_2({\bf r}))$, depends on the timing of the MRI sequence through the time evolution of the Bloch equation.
In this way, images with various desirable properties such as soft-tissue contrast can be obtained. The image contrast is based on differences in the
proton density, $M_0$, and/or the relaxation times, $T_1,T_2$, in biological tissue components; this unique feature provides a variety of medical
imaging applications and it is the cause for the success of MRI in medicine.

%%%%%%%%%%%%%%
\subsection{Selective excitation}
%{\it Selective \medskip excitation}\\
Multidimensional spatially selective excitation \cite{pauly} is an important concept of growing interest in MRI, e.g. in the field
of in vivo spectroscopy or for the challenging task of correcting subject-induced $B_1$ field inhomogeneities at ultra high fields \cite{ibrahim}.
However, thus far the computation of these pulses is based on a simplified physical model neglecting relaxation and
off-resonance effects during the pulse. Then, the calculation of the unknown RF pulse shape, $B_1$, from a
given target excitation pattern, $M_p$, reduces to a linear system:
\begin{equation}\label{eq:selexcommon}
 M_{p}({\bf r})\;=\;\mbox{i}\gamma M_0({\bf r})\int_{0}^{T}\,B_1(t)\exp[\mbox{i}{\bf r}\cdot{\bf k}(t)]\;dt
\end{equation}
where $M_0$ is the equilibrium magnetisation, ${\bf k}(t)$ is a given k-space trajectory.
A spatially and temporally discrete version of Eq.~(\ref{eq:selexcommon}) can be solved for $B_1$ by suitable generalised matrix inversion
methods.
Instead, the present approach provides a numerical method to design selective pulses under realistic experimental
conditions. The simulator computes the effective transverse magnetisation, $M_e({\bf r},t)$, which is used to correct the RF
pulse in order to account for effects not governed by Eq.~(\ref{eq:selexcommon}). Thus, a minimisation problem is formulated and
individually solved for all time steps $n\Delta t\;(n=1,…,N)$ , where $N\Delta t=T$ equals the pulse length:
\begin{equation}\label{eq:selexnew}
 || M_{p}({\bf r})\,-\, M_e[M_0({\bf r}),T_1({\bf r}),T_2({\bf r}),\Delta\omega({\bf r}),B_1(t),{\bf k}(t)]\,||\;=\;\min_{B_1}
\end{equation}
Here, the difference between the desired magnetisation pattern and the effective magnetisation pattern is minimised
with respect to the imaginary and the real part of $B_1= B_{1x} + \mbox{i}B_{1y}$. The starting point ($B_{1x},B_{1y}$) for each of
the $N$ consecutive 2D minimisation problems is taken from the solution of Eq.~(\ref{eq:selexcommon}). Note that the temporal sampling
of $B_1$ is taken from the discrete version of Eq.~(\ref{eq:selexcommon}), whereas the time evolution of the effective magnetisation
is computed with much higher accuracy by the simulator, i.e. within each interval $\Delta t$ the Bloch equation is individually solved
for each spin to compute the norm in Eq.~(\ref{eq:selexnew}) for the minimisation routine. Once a minimum is found for the n-th step of
the RF pulse, the final magnetisation states are taken as the starting condition for the next step.
%%%%%%%%%%%%%%%%%%%%%%%%%%%%%%%%%%%%%%%%%%%%%%%%%%%%%%%%
\section{Software Engineering}
\subsection{General design of the framework}
%{\it General design of the \medskip framework}\\
%%%%%%%%%%%%%
\epsfigure{fig/jemris-classes.eps}{Class hierarchy of the basic JEMRIS components, Sample, Signal, Model, and Sequence class,
                            respectively.  The PulseShape class has a strong link to the Sequence class.
                            (See text for description.)}{fig:classes}{bp}{1.0}
Modular and encapsulated development of software projects highly benefit from an object-oriented approach. However,
efficient implementations of numerical demanding problems are still dominated by sequential programming languages like
FORTRAN and C. For the purpose of the present MRI simulator which is developed in C++, an efficient numerical integrator
of the Bloch equation as well as an mechanism for transferring data in an distributed multi-processor environment was needed.
The former was found in the CVODE library \cite{cohen}, a highly efficient variable time stepping integrator for ordinary
differential equations. CVODE is written in C and therefore straight forward for integration in a C++ software project.
Message passing is realised with the C++ bindings of the MPI\footnote{The MPI standard. {\sf http://www-unix.mcs.anl.gov/mpi/}} library.
The software design of JEMRIS is based on the definition of four main classes: {\it Sample}, {\it Signal}, {\it Model}, and
{\it Sequence}, respectively. The {\it Sample} class describe the physical properties of the object, currently defined by the set
$P=(M_0,T_1,T_2)$ at every spatial position ${\bf r}=(x,y,z)$ of the sample.
Since MPI has no functionality for object serialisation, send/receive of sub-samples is realised with appropriate MPI data-types.
Similar, the {\it Signal} holds information on the `MR signal' consisting of the net magnetisation vector
${\bf M}(t)=[M_x(t),M_y(t),M_t(t)]$ at every sampled time point. MPI functionality is implemented in the same way as
for the {\it Sample} class. The {\it Model} class has the functionality for solving the physical problem.
The {\it Sequence} class is the most complex part of the framework: it is represented as a left-right ordered tree.
Fig.~\ref{fig:epitree} depicts by example how the different modules of an EPI sequence, loops and pulses, can be arranged in
an ordered tree. Trees can be effectively accessed by recursion and they are very well suited for building as well as accessing
values of the sequence. For this, a new object-oriented design pattern for MRI sequences is introduced which derives the basic parts,
loops (branch nodes) and modules (leaf nodes), from an abstract sequence class. The modules play out the pulses and, therefore,
have the functionality to set pulses of various types. The pulses themselves are defined in a separate class hierarchy.
JEMRIS uses an XML representation of the sequence tree, i.e. sequences can be read from or written in XML files. The C++ Xerces XML parser
provided by the Apache Software Foundation\footnote{\sf http://xml.apache.org/xerces-c/} is used for serialising or deserialising the sequence 
object. Further, a GUI was implemented in MATLAB\footnote{\sf http://www.mathworks.com} which allows to interactively build
the sequence tree, view the pulse diagram, and to perform the simulation. For the latter, MATLAB calls JEMRIS via {\sf ssh} on the remote HPC cluster.

\epsfiguresv{fig/EPI_sketch.eps}{fig//EPI_tree.eps}{{\bf Top:} Sketch of a native EPI pulse sequence diagram,
                consisting of an outer loop (e.g.~slices) in which
		the RF excitation module, the dephasing gradients, dead-times, and the inner loop for the EPI readout are nested.
		{\bf Bottom:} Representation of the same diagram with a left-right ordered tree. Branching nodes represent loops
		in which other modules are concatenated. Leaf nodes represent th modules in which the actual pulse shapes are
		played out.}{fig:epitree}{tp}{0.85}
%%%%%%%%%%%%%%%%%%%%%%%%%%%%%%%%%%%%%%%%%%%%%%%%%%%%
\begin{algorithm}[t]
 \caption{$\;\;${\footnotesize\sl Simulation Routine}\label{parcode1}}
 \ \\ \scriptsize initialise N+1 MPI processes (n=0,\dots,N)\\ 
 \raisebox{6cm}{\scriptsize
 \begin{minipage}[t]{5.5cm}
 	\ \\
 	{ MASTER process (p=0)}
 	\begin{algorithmic}[1]
  	\STATE	parse simulation XML file \\
	\STATE	instantiate {\it Sample} object\\
  	\STATE split {\it Sample} into $N$ {\it SubSample} objects\\
	\FOR{n=1 to N}
  	\SEND n-th {\it SubSample} to n-th slave \hfill$\longrightarrow$ 
  	\ENDFOR\\
  	\ \\
  	\ \\
  	\ \\
  	\ \\
  	\ \\
  	\ \\
  	\ \\
  	\ \\
  	\ \\
  	\ \\
  	\ \\
  	\ \\
  	\ \\
  	\ \\
  	\ \\
  	\ \\
  	\ \\
  	\ \\
  	\ \\
        \STATE instantiate {\it Signal}\\
	\FOR{n=1 to N}
  	\RECEIVE {\it SubSignal} from n-th slave \hfill$\longleftarrow$ 
  	\STATE {\it Signal} += {\it SubSignal(n)}\\
  	\ENDFOR\\
  	\STATE save {\it Signal} \\
  	\ \\
 	\end{algorithmic}
	\end{minipage}\hfill
	\begin{minipage}[t]{6cm}
 	\ \\
 	{SLAVE processes (p=1,\dots,N)}
 	\begin{algorithmic}[1]
  	\STATE	parse simulation XML file \\
	\STATE	instantiate {\it Model} object\\
  	\STATE	parse sequence XML file \\
	\STATE	instantiate {\it Sequence} object\\
  	\RECEIVE {\it SubSample}\\
  	\ \\
  	{\tiny\sf - functionality of {\it Model}::{\it Solve()} -}
  	\ \\
        \STATE instantiate {\it SubSignal}\\
        \STATE $N_s$ = {\it SubSample}::{\it getNumOfSpins()}
	\FOR{$s=1$ to $N_s$}
        \STATE instantiate {\it Spin(s)}\\
        \STATE {\it Sequence}::{\it Prepare()}
        \STATE $N_r$ = {\it Sequence}::{\it getNumOfRepetitions()}
        \STATE $N_c$ = {\it Sequence}::{\it getNumOfChildren()}
	\FOR{$r=1$ to $N_r$}
	\FOR{$c=1$ to $N_c$}
	\IF{ $N_c>0$ {\tiny\sf ( case {\it ConcatSequence} )} }
        \STATE {\it Sequence}={\it Sequence.getChild(k)}
        \STATE go to line {\footnotesize 10:} {{\tiny\sf - recursion -}}
	\ELSE[compute solution]
        \STATE instantiate CVODE solver
        \STATE obtain {\it Solution}
	\ENDIF
  	\ENDFOR\\
  	\ENDFOR\\
  	\STATE {\it SubSignal} += {\it Solution}\\
  	\ENDFOR\\
        \SEND {\it SubSignal} 
  	\ \\
  	\ \\
  	\ \\
 	\end{algorithmic}
 \end{minipage}\hfill
 }
\end{algorithm}
%%%%%%%%%%%%%%%%%%%%%%%%%%
\begin{algorithm}[h]
 \caption{$\;\;${\footnotesize\sl Selective Excitation Routine}\label{parcode2}}
 \ \\ \scriptsize initialise N+1 MPI processes (n=0,\dots,N)\\ 
 \raisebox{6cm}{\scriptsize
	%%% MASTER %%%
        \begin{minipage}[t]{5.5cm}
	\ \\
 	{MASTER process (p=0)}
 	\begin{algorithmic}[1]
  	\STATE	parse selective excitation XML file \\
	\ \\
  	\STATE split and {\bf send} {\it Sample} and {\it Target}  \hfill$\longrightarrow$ \\
  	  \ \\
  	- loop over timesteps -\\
	\FOR{$t=1$ to $N_t$}
        \STATE bool {\it bCont} = true \\
	\ \\
  	  {\tiny\sf - 2D conjugate gradient search for $B_1(t)$ -}\\
	  \WHILE{ {\it bCont} }
	    \FOR{$n=1$ to $N$}
  	      \RECEIVE $\varepsilon_n$ from n-th slave \hfill$\longleftarrow$ 
  	    \ENDFOR\\
	    \STATE {\it bCont} =  $[\sum \varepsilon_n \neq$ min$]$\\
  	    \STATE select next $B_1$ of the gradient search
  	    \STATE {\bf broadcast} $(B_{1x},B_{1y})$ and {\it bCont} \hfill$\longrightarrow$ \\
  	  \ENDWHILE\\
  	\ \\
  	  \STATE store final $B_1(t)$ \\
  	\ \\
  	\ENDFOR\\
  	\ \\
 	\end{algorithmic}
	%%% SLAVES %%%
	\end{minipage}\hfill
	\begin{minipage}[t]{6cm}
 	\ \\
 	{SLAVE processes (p=1,\dots,N)}
 	\begin{algorithmic}[1]
  	\STATE	parse selective excitation XML file \\
	\STATE	instantiate {\it Sequence} and {\it  Model} \\
  	\RECEIVE {\it SubSample} and {\it SubTarget} \\
  	  \ \\
	\FOR{t=1 to $N_t$}
        \STATE bool {\it bCont} = true \\
  	  \ \\
  	  {\tiny\sf - repeatedly call simulation routine -}
	  \WHILE{ {\it bCont} }
	     \STATE {\it Model::Solve()} in time interval $[t-1,t]$\\
  	     {\tiny\sf - send difference between target and excited magnetisation -}
	     \SEND $\varepsilon_n=\sum |M_p(x_i)-M_e(x_i,t)|$\\
  	     \ \\
  	     \ \\
  	     \ \\
  	     \RECEIVE $(B_{1x},B_{1y})$ and {\it bCont} \\
	     \STATE {\it Sequence.getChild(t)::setB1}($B_{1x},B_{1y}$)
  	  \ENDWHILE\\
  	  \ \\
  	\ENDFOR\\
  	\ \\
 	\end{algorithmic}
 \end{minipage}\hfill
 }
\end{algorithm}
%%%%%%%%%%%%%%%%%%%%%%%%%%
\subsection{Parallel implementation}
Pseudo code of the parallel workflow of JEMRIS is shown in the parallel algorithm diagram \ref{parcode1}. (Comments
are written with a reduced fontsize.) The (mostly sleeping) master process subdivides the problem
in the beginning and harvests the result in the end. The basic functionality of the slave processes is hidden
in the {\it Solve()} method of the {\it Model} class, where the solution for each spin is obtained during a recursive
travel through the sequence tree. All functionality on introducing off-resonance effects is hidden in the
lines {\footnotesize 9} and {\footnotesize 19} of the slave process. Settings about this functionality, as well the
setting of the sample, is provided in the simulation XML file, which is parsed in the \medskip beginning.\\
%%%
The computation of selective excitation pulses utilising the simulator is shown in the parallel algorithm diagram \ref{parcode2}.
The input parameters for the discrete problem consisting of $N_t$ steps and the appendant solution of the linear problem  
in Eq.~(\ref{eq:selexcommon}) is read from an XML file. After splitting the sample and the target pattern,
the slave processes instantiate a sequence object which consists of a root node and $N_t$ child nodes, each representing
one time segment with a constant RF pulse $(B_{1x},B_{1y})$. Then, within a loop over the time steps, the master solves
a conjugate gradient search to minimise the difference between the target pattern and the excited magnetisation by varying 
$(B_{1x},B_{1y})$. Thus, each slave repeatedly calculates its contribution to this difference each time with a new RF pulse
at the current segment. Once a minimum is found, the master stores the result and continues with the next time step at which the
slaves proceed with the final mangetisation state of the previous step.
%%%%%%%%%%%%%%%%%%%%%%%%%%%%%%%%%%%%%%%%%%%%%%%%%%%%%%%%
\section{Results}
\subsection{Benchmarks}
The simulations were performed on a 16 dual-core CPU Opteron cluster. This can be seen as small-scale HPC, however, it is sufficient
to perform 2D MRI simulations within minutes. Thus, the here presented simulations reduce to 2D examples, though the simulator
is also able to treat 3D simulations. The performance of the simulator is depicted in Fig.~\ref{fig:performance}. The results from
sequential program profiling show, that most of the time is spent in the highly optimised CVODE libraries, however, a significant
amount of 28 \% is spent in the {\it Sequence.getValue()} function, leaving room for future optimisation of the data retrieval from
the sequence tree. As expected, the parallel performance scales nearly perfectly, i.e.~the speed increases linearly with the number
of processors.
%%%%%%%%%%%%%%%%
\epsfigure{fig/performance.eps}{JEMRIS performance testing. {\bf Left:} Sequential profiling shows that $\approx$ 70 \% of the computing time
                             is spent in the highly optimised CVODE library. {\bf Right:} The speed
			     gain due to parallelisation (right) is close to optimal for large scale problems, as expected.}
			{fig:performance}{bp}{1.0}
\subsection{MRI simulation examples}
\epsfigures{fig/GUI_seq_ex1.eps}{fig/GUI_sim_ex1.eps}{{\bf Left:}  Example of the JEMRIS GUI for sequence development: the EPI sequence
			     already shown in Fig.~\ref{fig:epitree}. In the top left the sequence is interactively build and on the right
			     the parameters of the individual nodes are set. In the bottom the corresponding sequence
			     diagram is shown. Possible error output is given in the slider-window.
                             {\bf Right:} Example of the JEMRIS GUI for simulations: a simple three pulse sequence
			     ( 60$^\circ$-90$^\circ$-90$^\circ$ at 0-10-25 ms) on a homogeneous sphere under strong field inhomogeneities
			     results in the famous five echoes: three spin echoes, one reflected spin echo, and the stimulated
			     echo.}{fig:gui}{tp}{0.45}
\epsfigure{fig/simartex.eps}{Example of artefact simulations on a human head model: {\bf a} EPI with chemical shift.
                               {\bf b} EPI distortions due to a nonlinear read gradient. {\bf c} TrueFISP banding artefacts
				resulting from susceptibility-induced field inhomogeneities. {\bf d} Artefact in a spin echo
				sequence with a long refocusing pulse in the order of $T_2$ of the
				sample.}{fig:artifacts}{tp}{1.0}
%%%%%%%%%%%%
Fig.~\ref{fig:gui} shows an example of the GUI for sequence development, which allows an interactive
building of the sequence tree and various representations of the corresponding pulse diagram. The right part of Fig.~\ref{fig:gui}
depicts the simulation GUI, showing an example of a simple 3-pulse experiment without any gradients involved, however, strong
random field variations are applied yielding a very strong $T_{2}^{*}$ effect, i.e.~signal decays rapidly to zero. The
famous five MR echoes generated by such a three pulse experiment, are visible. Note that the tracking of magnetisation can be accurately
performed at any desired time scale also during the application of RF pulses, since a generally valid numerical solution is computed.
Examples of MRI artefact generation are given in Fig.~\ref{fig:artifacts}: a) EPI simulation considering chemical shift effects
yield the prominent fat-water shift in MRI; b) Malfunction MR scanner hardware simulation with a nonlinear gradient field results
in a distorted EPI image; c) TrueFISP simulation including susceptibility-induced field inhomogeneity yield the well-known
banding artefacts in the human brain; d) artefact in spin echo imaging due to a very long inversion pulse exciting transverse
magnetisation. Note that the latter example can not be realised with any simulator relying on analytical solutions of the Bloch
equation due to neglection of relaxation effects during the RF pulse. The exact simulation of the simultaneous occurrence of RF
pulses and imaging gradients with JEMRIS is the basement for the derivation of new selective excitation pulses.
\begin{small}
 \begin{table}[tp]
  {\footnotesize
  \begin{center}
  \begin{tabular}{|c|c|c|c|}
  \hline
        Figure & Number of Spins & Sequence & calculation time [min] \\
  \hline
	\ref{fig:gui} (right)    & 10.000     & ThreePulses & 0.1 \\
	\ref{fig:artifacts} a),b)& 60.000     & EPI         & 1 \\
	\ref{fig:artifacts} c)   & 60.000     & TrueFISP    & 10 \\
	\ref{fig:artifacts} d)   & 60.000     & Spin Echo   & 10 \\
  \hline
  \end{tabular}
  \end{center}
  }
  \caption{\label{tab:calctimes}
           {\it Calculation times of the simulation examples. Note, that the calculation time strongly depends
		on the type of imaging sequence.}}
 \end{table}
\end{small}
The calculation times for the shown examples are given in Table \ref{tab:calctimes}.
%%%%%%%%%%%%
\epsfigure{fig/Selex.eps}{Example of multidimensional excitation. {\bf Left Column:} 2D MR sample and the target pattern for selective
			     excitation. {\bf Middle Column:} HF pulses computed with the common approach and with the newly proposed method,
			     based on the JEMRIS framework. {\bf Right Column:} Simulation results, showing the excited spin
			     system after applying the corresponding HF pulses.}{fig:selex}{bp}{0.7}
\subsection{RF pulse design for selective excitation}
For the demonstration of selective excitation, a homogeneous spherical object and a desired target
magnetisation pattern were defined and the RF pulses of the common model and the new model were computed
according to Eq.~(\ref{eq:selexcommon}) and (\ref{eq:selexnew}), respectively. Simulations were performed for $\approx$ 30,000 spins.
The results are depicted in Fig.~\ref{fig:selex}. Note that the excitation patterns are not the result of a (simulated) imaging sequence
but they are the effective patterns of excited spins directly after the pulse. In comparison, the optimised RF pulse
sequence computed with the new approach shows less high spatial frequency components as with the common
approach. This is a well-known problem of the common linear approach, leading to high frequency artefacts on
selectively excited images, as also seen from the excited target pattern in Fig.~\ref{fig:selex}. In comparison, the new approach
excites a more homogeneous target pattern.
%%%%%%%%%%%%%%%%%%%%%%%%%%%%%%%%%%%%%%%%%%%%%%%%%%%%%%%%%%%%%%%%%%%%%%%%%%%%%%%%%%%%%%%%%%%%%%%%%%%
\section{Conclusion}
JEMRIS performs (classically) general MRI simulations. It allows time tracking of the net magnetisation, separation
of different effects, and therefore accurate investigation of MR samples (e.g. implants) and MRI sequences under a
variety of experimental conditions. The parallel HPC implementation allows the treatment of huge spin ensembles
resulting in realistic MR signals. On small HPC clusters, 2D simulations can be obtained in the order of minutes,
whereas for 3D simulations it is recommended to use higher scaling HPC systems. As shown, the simulator can be
applied to MRI research for the development of new pulse sequences. The impact of these newly designed pulses in real
experiments has to be tested in the near future.
Further, the rapidly evolving field of medical image processing might benefit of a “gold standard”, serving as a testbed
for the application of new methods and algorithms. Here, JEMRIS could provide a general framework for generating
standardised and realistic MR images for many different situations and purposes. This aspect is especially interesting,
since the image processing community usually has adequate access to HPC systems.

\begin{thebibliography}{10}
\bibitem{benoit}
H.~Benoit-Cattin, G.~Collewet, B.~Belaroussi, H.~Saint-Jalmes, and C.~Odet,
  ``The {SIMRI} project: a versatile and interactive {MRI} simulator,''
  \emph{Journal of Magnetic Resonance}, vol. 173 (1), pp. 97--115, 2005.

\bibitem{haacke}
E.M.~Haacke, R.W.~Brown, M.R.~Thompson, and R.~Venkatesan, ``Magnetic Resonance Imaging: Physical Principles and Sequence Design,''
  \emph{Wiley \& Sons}, 1999.

\bibitem{cohen}
S.~Cohen and A.~Hindmarsh, ``{CVODE}, a stiff/nonstiff {ODE} solver in {C},''
  \emph{Computers in Physics}, vol. 10 (2), pp. 138--143, 1996.

\bibitem{pauly}
J.~Pauly, D.~Nishimura, and A.~Macovski, ``A k-space analysis of small tip-angle excitation pulses,''
  \emph{Journal of Magnetic Resonance}, vol. 81, pp. 43--56, 1989.

\bibitem{ibrahim}
T.~Ibrahim, R.~Lee, A.~Abduljalil, B.~Baertlein, and P.~Robitaille,
  ``Dielectric resonances and $B_1$ field inhomogeneity in {UHFMRI}:
  computational analysis and experimental findings.,''
  \emph{Magnetic Resonance Imaging}, vol. 19(2), pp. 219--226, 2001.

\end{thebibliography}

\end{document} 
