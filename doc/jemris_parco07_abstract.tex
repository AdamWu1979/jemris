\documentclass{nic-series}

\begin{document} 

\title{HPC Simulation of Magnetic Resonance Imaging}

\author{Tony~St\"ocker, Kaveh Vahedipour, \and N.~Jon~Shah }

\institute{Institute of Medicine,\\
         Research Centre J\"ulich, 52425 J\"ulich, Germany\\
         \email{\{t.stoecker, k.vahedipour, n.j.shah\}@fz-juelich.de}
          }

\maketitle

\section*{Abstract}
High performance computer (HPC) simulations provide helpful insights to the process of magnetic resonance
image (MRI) generation,  e.g. for general MRI pulse sequence design \cite{haacke} and optimization, MRI artefact detection,
validation of experimental results, hardware optimization, MRI sample development and for education purposes. This abstract presents
the recently developed simulator JEMRIS (J\"ulich Environment for Magnetic Resonance Imaging Simulation). JEMRIS is
developed in C++ and the message passing is realised with the MPI library. In contrast to previous approaches known from the literature,
JEMRIS provides generally valid numerical solutions for the case of classical MR spin physics governed by the Bloch equations.\\
A new object-oriented design pattern for the rapid implementation of MRI
sequences was developed. Here the sequence is defined by a set of interacting objects inserted into a left-right ordered tree structure. 
The framework provides a GUI for the rapid development of the MRI sequence tree, which is stored in XML format and parsed into a
C++ object for the simulation routines. Thus, the set-up of MR imaging sequences relying on the pre-defined modules is easily performed
without any programming.\\
The parallel HPC implementation allows the treatment of huge spin ensembles
resulting in realistic MR signals. On small HPC clusters, 2D simulations can be obtained in the order of minutes,
whereas for 3D simulations it is recommended to use higher scaling HPC systems. The simulator can be
applied to MRI research for the development of new pulse sequences.
The framework already serves as a tool in research projects, as will be shown for the important example of
multidimensional spatially selective excitation.
The impact of these newly designed pulses in real experiments has to be tested in the near future.
Further, the rapidly evolving field of medical image processing might benefit of a “gold standard”, serving as a testbed
for the application of new methods and algorithms. Here, JEMRIS could provide a general framework for generating
standardised and realistic MR images for many different situations and purposes.

\begin{thebibliography}{10}

\bibitem{haacke}
E.M.~Haacke, R.W.~Brown, M.R.~Thompson, and R.~Venkatesan, ``Magnetic Resonance Imaging: Physical Principles and Sequence Design,''
  \emph{Wiley \& Sons}, 1999.
\end{thebibliography}

\end{document} 
